%% LyX 2.2.2 created this file.  For more info, see http://www.lyx.org/.
%% Do not edit unless you really know what you are doing.
\documentclass[english]{article}
\usepackage[T1]{fontenc}
\usepackage[latin9]{inputenc}
\usepackage[a4paper]{geometry}
\geometry{verbose,tmargin=3cm,bmargin=2.5cm,lmargin=2.5cm,rmargin=2.5cm}
\usepackage{color}
\definecolor{note_fontcolor}{rgb}{0.800781, 0.800781, 0.800781}
\usepackage{array}
\usepackage{verbatim}
\usepackage{rotfloat}
\usepackage{url}
\usepackage{amstext}
\usepackage{amsthm}
\usepackage{graphicx}
\usepackage{setspace}
\doublespacing

\makeatletter

%%%%%%%%%%%%%%%%%%%%%%%%%%%%%% LyX specific LaTeX commands.
%% Because html converters don't know tabularnewline
\providecommand{\tabularnewline}{\\}
%% The greyedout annotation environment
\newenvironment{lyxgreyedout}
  {\textcolor{note_fontcolor}\bgroup\ignorespaces}
  {\ignorespacesafterend\egroup}

%%%%%%%%%%%%%%%%%%%%%%%%%%%%%% Textclass specific LaTeX commands.
\theoremstyle{plain}
\newtheorem{thm}{\protect\theoremname}
\ifx\proof\undefined
\newenvironment{proof}[1][\protect\proofname]{\par
\normalfont\topsep6\p@\@plus6\p@\relax
\trivlist
\itemindent\parindent
\item[\hskip\labelsep\scshape #1]\ignorespaces
}{%
\endtrivlist\@endpefalse
}
\providecommand{\proofname}{Proof}
\fi

%%%%%%%%%%%%%%%%%%%%%%%%%%%%%% User specified LaTeX commands.
\usepackage{amsmath,epsfig,natbib,pstricks,pst-text,pst-node,pst-plot,amssymb,wrapfig,threeparttable,rotating}

\makeatother

\usepackage{babel}
\providecommand{\theoremname}{Theorem}

\begin{document}

\title{WST 312 Template}

\author{First Name and Surname Student Number\vspace{0.5cm}\\
Department of Statistics, University of Pretoria\vspace{0.5cm}\\
\includegraphics[width=5cm]{UPlogohighres.jpg}\vspace{0.5cm}\\
Enter date of document here}

\date{}

\maketitle
\newpage{}

\tableofcontents{}

\listoffigures

\listoftables

\newpage{}

\section{Equation Formats}

You can rename the section to what ever is approriate.

Inline equation: $x=y$

Display equation: \[x=y\]

Numbered display equation: 
\begin{equation}
x=y\label{eq:afe,fjgd}
\end{equation}



Equation array: 

\begin{eqnarray*}
x & = & y+2+3\\
 & = & y+5.
\end{eqnarray*}

Equation array:

$f(x)=\left\{ \begin{array}{cc}
1 & \textrm{\textrm{if }}x<2\\
2 & \textrm{if }x\ge2
\end{array}\right.$ {and}
$f(x)=\left\{ \begin{array}{cc}
1 & \textrm{\textrm{if }}x<2\\
2 & \textrm{if }x\ge2
\end{array}\right.$


\section{Tables and Figures}

In Figure \ref{fig:UPlogo} and in Figure \ref{fig:Caption} we see
the UP logo. %

\begin{figure}[h]
\centering
\includegraphics[width=6cm]{UPlogohighres.jpg}
\caption{A graphic of the UP logo\label{fig:UPlogo}}
\end{figure}




\begin{sidewaystable}
\begin{centering}
\begin{tabular}{|>{\centering}p{3cm}|c|c|c|>{\centering}p{3cm}|}
\multicolumn{1}{>{\centering}p{3cm}|}{} &  &  &  & \tabularnewline
klvunl ruvynryrinfsfgng hdfhfgjf gjfhjghjhk jhkghjkh dfhdj ghkjfgj
djdj dghjhhjghj & aevirue$x=y$n oiurlevniq & ermwovmi;q & aevpimqw & ovmi;\tabularnewline
 &  &  &  & \tabularnewline
 &  &  &  & \tabularnewline
 &  &  &  & \tabularnewline
\end{tabular}
\par\end{centering}
\caption{rvybruvybrv \label{tab:test}}

\end{sidewaystable}

\begin{table}
\begin{centering}
\begin{tabular}{|>{\centering}p{4cm}|>{\centering}p{4cm}|>{\centering}p{4cm}|}
\hline 
\multicolumn{1}{|>{\centering}p{4cm}||}{} & \multicolumn{1}{>{\centering}p{4cm}||}{Column 1} & Column 2\tabularnewline
\hline 
\hline 
Row 1 &  & \tabularnewline
\hline 
Row 2 &  & \tabularnewline
\hline 
\end{tabular}
\par\end{centering}
\caption{Example \label{tab:Example}}
\end{table}
 The application should be presented in this section. Code should
be included in an appendix as well as additional output if needed. 

\newpage{}

\bibliographystyle{plain}
\bibliography{BibliographyDatabase}

\newpage{}

\section*{Appendix\addcontentsline{toc}{section}{Appendix}}

Include any additional code, output or data here. 
\end{document}
